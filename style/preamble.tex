%%%%%%%%% Extra preamble info %%%%%%%%%%%%%%%

% ---- CUSTOM AMPERSAND
% \newcommand{\amper}{{\fontspec[Scale=.95]{Adobe Caslon Pro}\selectfont\itshape\&}}


% optional T1 font encoding for journal
\usepackage[T1]{fontenc}

% *** MATH PACKAGES ***
%
\usepackage{amsmath}
% A popular package from the American Mathematical Society that provides
% many useful and powerful commands for dealing with mathematics.
% Do NOT use the amsbsy package under comsoc mode as that feature is
% already built into the Times Math font (newtxmath, mathtime, etc.).
% 
% Also, note that the amsmath package sets \interdisplaylinepenalty to 10000
% thus preventing page breaks from occurring within multiline equations. Use:
\interdisplaylinepenalty=2500
% after loading amsmath to restore such page breaks as IEEEtran.cls normally
% does. amsmath.sty is already installed on most LaTeX systems. The latest
% version and documentation can be obtained at:
% http://www.ctan.org/pkg/amsmath


% Select a Times math font under comsoc mode or else one will automatically
% be selected for you at the document start. This is required as Communications
% Society journals use a Times, not Computer Modern, math font.
\usepackage[cmintegrals]{newtxmath}
% The freely available newtxmath package was written by Michael Sharpe and
% provides a feature rich Times math font. The cmintegrals option, which is
% the default under IEEEtran, is needed to get the correct style integral
% symbols used in Communications Society journals. Version 1.451, July 28,
% 2015 or later is recommended. Also, do *not* load the newtxtext.sty package
% as doing so would alter the main text font.
% http://www.ctan.org/pkg/newtx

% Bold math
\usepackage{bm}


% Set figure legends and captions to be smaller sized sans serif font
% \usepackage[font=\tiny]{caption}
\usepackage[font=footnotesize]{caption}

% reduce space between text and floats
\addtolength{\textfloatsep}{-0.5cm}

\usepackage{siunitx}

% Don't let floats cross subsections
% \usepackage[section]{tex_packages/extraplaceins}

% pandoc-crossref defines (overridden by this file, so adding back here)
\usepackage{cleveref}
\crefname{figure}{Fig.}{Figs.}          % Label format for 'figure' env.
\Crefname{figure}{Figure}{Figures}      % Start-of-sentence variant
\crefname{section}{Sec.}{Secs.}
\Crefname{section}{Section}{Sections}
\crefname{algorithm}{Alg.}{Algs.}
\Crefname{algorithm}{Algorithm}{Algorithms}
\crefname{equation}{Eq.}{Eqs.}
\Crefname{equation}{Equation}{Equations}

% Small bib
\AtBeginBibliography{\small}

\usepackage{algorithm} % <- from the `algorithms` package
\usepackage{algorithmicx}
\usepackage{algpseudocode}
\algnewcommand\algorithmicinput{\textbf{Input:}}
\algnewcommand\Input{\item[\algorithmicinput]}
\algnewcommand\algorithmicoutput{\textbf{Output:}}
\algnewcommand\Output{\item[\algorithmicoutput]}
\algnewcommand\algorithmicto{\textbf{to}}

% Subfigures
\usepackage{subfig}

% IEEE options
% \usepackage[letterpaper, left=1in, right=1in, bottom=1in, top=0.75in]{geometry}
%\usepackage{fancyhdr}
%\usepackage{graphicx}
%\usepackage{psfrag}
%\usepackage{subfigure}
%\usepackage{url}
%\usepackage{stfloats}
%\usepackage{amsmath}
%\usepackage{array}
%\usepackage{fancyhdr}
%\usepackage{epsfig}
%\usepackage{amssymb}
%\usepackage{color}

